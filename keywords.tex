\section{List of keywords}
This section contains a rather extensive list of all the keywords. 
Most of the keywords have default values, which are given in the 2nd column. 
Some of the keywords may be rather obscure and related only to technicalities or special applications.
If you feel that you do not understand some keyword, don't worry, it's probably not important.
The keywords that you most likely don't need to care about are given with \shade{grey background}.
Note that the program will sometimes prevent you from using experimental features, weird keyword combinations
or debug options unless you set $ iknow=1 $ in namelist "general". It also tries to sanitize user input against obvious errors.
But do not rely on this feature! You can still make your simulation meaningless by choosing bad simulation parameters.

The keywords are divided into namelists that they belong to.
The namelists \textit{general} and \textit{nhc} must always be present in the input file.
Order of namelists in the input file is irrelevant.

% We should color whole cell, not only keyword, which looks ugly. See:
% http://tex.stackexchange.com/questions/50349/color-only-a-cell-of-a-table

\subsection{\&general}
\begin{tabularx}{\textwidth}{lcX}
\textit{keyword} & \textbf{default} & Description \\
\hline 

\textit{pot} & - & Specifies where do we get forces and energies. Supported options are: \\
 &  & \textbf{PROGRAM} --- \abin will look for BASH interface in file \textit{PROGRAM/r.program}
  i.e. ORCA/r.orca \\
 &  & \textbf{nab} --- AMBER force field\\ 
 &  & \textbf{mmwater} --- force field for water (see keyword \textit{watpot} ) \\
 &  & \textbf{harm} --- diatomic harmonic oscillator \\ 
 &  & \textbf{morse} --- Morse potential for diatomic molecule\\
 &  & \textbf{2dho} --- \textit{n}-dimensional one-atom harmonic oscillator (n=1-3) \\ 

\textit{natom} & -- &  Number of atoms. \\ 
 
\textit{pimd} & \textbf{0} & Controls the type of simulation. \\  
 & & \textbf{0} --- Classical simulation \\ 
 & & \textbf{1} --- Path-integral simulation \\  
 & & \textbf{2} --- Surface Hopping \\
 & & \textbf{3} --- Simple steepest-descent minimization \\ 

\textit{nstep} & \textbf{1}& Number of time steps. \\

\textit{dt} & \textbf{20} & Time step [a.\,u.] \\

\textit{irandom} & \textbf{156873} & Random seed \\ 

\textit{irest} & \textbf{0} & Controls the restart of the simulation. \\
 & & \textbf{0} -- Do not restart. Take the initial structure from mini.dat \\
 & & If restart.xyz is present, end with and error for safety reasons. \\
 & & \textbf{1} -- Restart the simulation from file "restart.xyz". \\ 
 & & Copy the file "restart.xyz" to "restart.xyz.\textit{it}", where \textit{it} is the current timestep.  \\

\textit{imini} & \textbf{0} &  Number of time steps for equilibration of the system. 
During this period, the coordinates are printed to file movie\_mini.xyz. Should be at least 2\,picoseconds for small simple systems. \\

\textit{conatom} & \textbf{0} & Number of frozen atoms. \\ 
& & \\
& & \textsc{Keywords for PIMD and PI-GLE simulations} \\
\textit{nwalk} & \textbf{1} & Number of random walkers (beads) for PIMD. \\
 & & It is automatically set to 1 if \textit{pimd=0}. \\
 
\textit{istage}  & \textbf{0} & Controls the coordinates used for PIMD.\\
 & & \textbf{0} -- Normal cartesian coordinates.  \\
 & & \textbf{1} -- Staging coordinates; these should we always used with PIMD. \\
\textit{nproc} & \textbf{1} & Number of processors for parallel simulations. \\
& & Should be 1 or equal to \textit{nwalk}. 	 \\
\colorbox{black!20}{\textit{nabin}} & \textbf{50} & Number of substeps in RESPA algorithm used for PIMD. \\
 
& & \\
& & \textsc{Options controlling the output} \\
\textit{nwrite} & \textbf{1}  & Print basic output every \textit{nwrite} step (e.g. temperature, energies).  \\
\textit{nwritex}& \textbf{1}  & Print coordinates every  \textit{nwritex} step. \\
\textit{nrest} & \textbf{1}   & Print restart file every \textit{nrest} step.   \\
& &  Larger value is recomended, since restart files can be quite big. \\
\textit{nwritev} & \textbf{0} & Print velocities every \textit{nwritev} step.   \\
\colorbox{black!20}{\textit{ncalc}} & \textbf{1}   & Calculate observables (energies, temperature) every \textit{ncalc} step. 	\\

\end{tabularx}
\newpage

\begin{tabularx}{\textwidth}{lcX}

& & \textsc{Other options} \\
\textit{isbc }&\textbf{0} & Controls spherical boundary conditions (SBC).\\
& & \textbf{0} -- Off.\\
& & \textbf{1} -- On.\\
\textit{rb\_sbc} & -- & Radius of the sphere for SBC in Angstroms.\\
& &  If not given, the radius of the initial structure is used. \\
\textit{kb\_sbc} & \textbf{0.02} &  Force constant for SBC in atomic units. \\

\colorbox{black!20}{\textit{anal\_ext}} & \textbf{0} & Should we call external analysis routine? \\
 & &User can write his/her own routine, which is called during analysis.\\
 & &  Template for this subroutine is in file analyze\_ext\_template.f90. \\

\colorbox{black!20}{\textit{icv}}  & \textbf{0} &	 Should we calculate constant volume heat capacities? \\
\colorbox{black!20}{\textit{ihess}}& \textbf{0} &  Should we compute hessian for heat capacities? \\
& &  Currently only implemented for pot=\textit{nab},\textit{harm} or \textit{morse}.   \\
\end{tabularx} 


\subsection{\&system}
\begin{tabularx}{\textwidth}{lcX}
\textit{keyword} & \textbf{default} & Description \\
\hline 
\textit{massnames} & \textit{array} & Array of atom names, for which we define user-specified masses. \\
\textit{masses} & \textit{array} &  Array of user-specified masses. \\
& & This array should match the \textit{massnames} array. \\
& & \\

& & \textsc{Shake options} \\
\textit{nshake} & \textbf{0} & Number of fixed bonds. \\
\textit{shake\_tol} & \textbf{0.001} & Shake convergence tolerance \\
\textit{ishake1} & \textit{array} & Array of first-atom indices specifying the fixed bonds.  \\
\textit{ishake2} & \textit{array} & Array of second-atom indices specifying the fixed bonds. \\
& & \\
& & \textsc{Options related to geometrical analyses} \\
\shade{\textit{ndist}   }& \textbf{0} & Number of distances for analysis. \\
\shade{\textit{nang} }& \textbf{0} & Number of angles for analysis. \\
\shade{\textit{ndih} }& \textbf{0} & Number of dihedral angles for analysis. \\

\shade{\textit{dist1} }& \textit{array} & Array of first-atom indices for bond definition.\\
\shade{\textit{dist2} }& \textit{array} & Array of second-atom indices for bond definition.\\
\shade{\textit{ang1} }& \textit{array} & Array of first-atom indices for angle definition.\\

\shade{\textit{xmin} }& 0.5 & Minimum bond distance for binning. \\
\shade{\textit{xmax} }& 5.0 & Maximum bond distance for binning. \\
\shade{\textit{disterror} }& \textbf{0} & If set to \textbf{1}, abort the simulation
if the bond distance is outside the range (\textit{xmin},\textit{xmax}).\\ 

\shade{\textit{ang2} }& \textit{array} & Array of second-atom indices for angle definition.\\
\shade{\textit{ang3} }& \textit{array} & Array of third-atom indices for angle definition.\\

\shade{\textit{dih1} }& \textit{array} & Array of first-atom indices for dihedral definition.\\
\shade{\textit{dih2} }& \textit{array} & Array of second-atom indices for dihedral definition.\\
\shade{\textit{dih3} }& \textit{array} & Array of third-atom indices for dihedral definition.\\
\shade{\textit{dih4} }& \textit{array} & Array of fourth-atom indices for dihedral definition.\\


\shade{ \textit{nbin} }& \textbf{2000} & Number of bins for the histograms of bond densities. \\
\shade{\textit{nbin\_ang}} & \textbf{180} & Number of bins for the histograms of angles. \\
& & \\

\hline
\end{tabularx}

\newpage
\subsection{\&nhcopt}
\begin{tabularx}{\textwidth}{lcX}
\textit{keyword} & \textbf{default} & Description \\
\hline
\textit{temp} & \textbf{0.0} &  Temperature in Kelvins.
This also controls the generation of initial velocities if \textit{temp0} is not explicitly set. \\ 
\textit{temp0} & -- &  Temperature of initial Maxvell-Boltzmann distribution. \\
\textit{inose} & \textbf{0} &  Controls the use of a thermostat. \\
& & \textbf{0} -- Microcanonical trajectory. No thermostat. \\
& & \textbf{1} -- Nos\'{e}-Hoover thermostat.\\
& & \textbf{2} -- Quantum GLE thermostat.\\


\textit{nchain} & \textbf{4} & Number of Nos\'{e}-Hoover chains. This should be always $>1$ for small systems. \\

\textit{tau0} & \textbf{0.001} & Relaxation time of the thermostat in picoseconds. From this parameter, 
the thermostat masses are determined automatically based on temperature and size of the system.\cite{Tuckerman_book} \\

\textit{imasst} & \textbf{1} & Determines, whether we use massive or global thermostat. \\
& & \textbf{0} -- Use global thermostat.
Requires further specification of the system using \textit{nmolt}, \textit{natmolt} and \textit{nshakemol} variables.
WARNING: This was not thoroughly tested.
Since we do not remove rotations, this thermostat can produce wrong results. It is however required if you use SHAKE.\\
& & \textbf{1} -- Use massive thermostat. Each degree of freedom has its own NH chain.
This is required for PIMD and recommended for classical simulations. \\

\textit{nmolt} & \textbf{1} & If we use global NHC thermostat, the system is divided into \textit{nmolt} parts
and each part has its own NHC thermostat. \\

\textit{natmolt} & \textit{array}& Array of integers specifying the number of atoms in each part that has its own thermostat.
This assumes, that the parts are consecutively aligned in the input coordinates.  \\

\textit{nshakemol} & \textbf{0} & Array of integers specifying a number of fixed bonds in each part of the system that has its own thermostat.
Note that you can not fix bonds between atoms from different parts of the system. \\

\colorbox{black!20}{nrespnose} & \textbf{3} & Number of sub-steps of the NHC integrator.\cite{Tuckerman_book} \\

\colorbox{black!20}{nyosh} & \textbf{7} & Number of weights in the Suzuki-Yoshida scheme.\cite{Tuckerman_book} \\
& & \textbf{1} - Do not use Suzuki-Yoshida scheme (not recommended) \\
& & 3 -- 4th order Suzuki-Yoshida scheme.\\
& & 7 -- 6th order Suzuki-Yoshida scheme.\\  

\colorbox{black!20}{\textit{ams}} & -- &  Nos\'{e}-Hoover mass.
This number is always chosen automatically for PIMD simulations.
This parameter overrides the \textit{tau0} parameter. \\

\colorbox{black!20}{\textit{scaleveloc}} & \textbf{1} & Whether we scale initial velocities so that the total kinetic energy
matches the desired temperature. This should reduce the initial oscillations of the thermostat. \\ 

\colorbox{black!20}{\textit{initnhc}} & \textbf{1} & Whether we initialize NHC momenta. \\ 

\colorbox{black!20}{\textit{readnhc}} & \textbf{1} & Read NHC momenta from restart.xyz if $inose=1$. \\ 

\end{tabularx}


\newpage
\subsection{\&sh}
This namelist contains parameters for surface hopping simulations and is read only if \textit{ipimd}=3.
 
\hspace*{-0.4cm} 
\begin{tabularx}{\textwidth}{lcX}
\textit{keyword} & \textbf{default} & Description \\
\hline
\textit{nstate} & \textbf{1} & Number of electronic adiabatic states.\\
%\rule[0ex]{0pt}{4ex}
\textit{istate\_init} & \textbf{1} & Initial electronic state. \\ 

\textit{deltae} & \textbf{5.0} & Energy threshold for the computation of NAC vectors.
The NAC vector between states is not computed if the difference in energy is $ > deltae$.\\ 
 
\textit{popthr} & \textbf{0.001} & The NAC vector between two states is not computed,
unless at least one of them has population $ > popthr$. Set this to some negative number to disable this feature. \\

\textit{nac\_accu1} & \textbf{7} & Default accuracy of NAC vector is $10^{-nac\_accu1}$. \\

\textit{nac\_accu2} & \textbf{5} & If the calculation of NAC does not converge, 
try with accuracy $10^{-nac\_accu2}$. \\


\textit{energydifthr} & \textbf{1.0} &  Abort the simulation if the total energy in two consecutive
time steps differs more than \textit{energydifthr}.       \\
 
\textit{energydriftthr} & \textbf{1.0} & Abort the simulation the total energy drifts 
from its initial value more than this threshold. 
This criterion is more strict than \textit{energydifthr}. \\

\textit{substep} & \textbf{100} & Number of substeps for the integration of electronic SE. \\

\shade{\textit{popsumthr}} & \textbf{0.001} & If the norm of the electronic wave function differs from 1
more than \textit{popthr}, the simulation is aborted.
You should probably increase \textit{substep}. \\

\shade{\textit{alpha}} & \textbf{0.1} & Empirical parameter for the decoherence correction, implemented according to reference \cite{Granucci2007}.
If you do not want to use decoherence correction, set $alpha <= 0 $. \\
   

\shade{\textit{integ}} & \textbf{butcher} & Integrator of the electronic Schr\"{o}dinger equation. \\
& & \textbf{euler} -- Euler integrator. Only for debugging purposes!\\
& & \textbf{rk4} -- Runge-Kutta 4th-order integrator. Not recommended.\\
& & \textbf{butcher} -- Butcher's 5th-order integrator. Default and recommended.\\

\shade{\textit{nohop}} & \textbf{0} & Hopping is forbidden if $nohop=1$. \\

\shade{\textit{inac}} & \textbf{0} & Controls whether we use NAC couplings or time-derivative couplings. \\
 & & \textbf{0} -- Use non-adiabatic couplings from \textit{ab initio} program. \\
 & & \textbf{1} -- Use time-derivative couplings. \\
 
\shade{\textit{adjmom}} & \textbf{0} & Controls the adjustment of velocity after the hopping event. \\
 & & \textbf{0} -- Adjust the velocity along the direction of NAC vector.
 If there is not enough kinetic energy in that direction,
 try simple scaling as if \textit{adjmom}=1. \\
 & & \textbf{1} -- Scale the velocity, if there is enough kinetic energy. \\
 & & This is the only and default option if \textit{inac}=1.  \\
 
 \shade{\textit{revmom}} & \textbf{0} & Should we reverse the velocity in case of frustrated hop? \\
 & & \textbf{0} -- No. \\
 & & \textbf{1} -- Yes. \\
 
 \shade{\textit{phase}} & \textbf{0} & Should we integrate the phase along the trajectory? \\
 & & \textbf{0} -- No. \\
 & & \textbf{1} -- Yes. \\
 
\end{tabularx}




%\subsection{\&qmmm}
%\begin{description}
%\item[natqm] Number of quantum atoms. These must be first in \textit{mini.dat}!
%\end{description}
%
%\subsection{\&nab}
%\begin{description}
%\item[ipbc] Control PBC.Must match conditions in input.top\newline
%\textbf{0} -- PBC off.\newline 
%\textbf{0} -- PBC on. 
%\end{description}



